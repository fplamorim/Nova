%!TEX root = ../thesis_rui_almeida.tex
%%%%%%%%%%%%%%%%%%%%%%%%%%%%%%%%%%%%%%%%%%%%%%%%%%%%%%%%%%%%%%%%%%%
%% chapter1.tex
%% Rui V. Almeida's thesis file
%%
%% Chapter with introduciton
%%%%%%%%%%%%%%%%%%%%%%%%%%%%%%%%%%%%%%%%%%%%%%%%%%%%%%%%%%%%%%%%%%%
\chapter{Introduction}%
\label{cha:introduction}

\section{Background, Motivation and Starting Points}%
\label{sec:background_and_motivation}

\subsection{Introduction}%
\label{sub:introduction}

This thesis describes the work that I have done in the past 4 years on
the design and development of a miniaturized system for atmospheric
monitoring based on optical spectroscopy. The project itself was the
major part of the \gls{ATMOS}, an initiative that was
contemplated with European funding through a \gls{PT2020} initiative and
came as a response to the growing weight that \gls{AP} has in the whole
Western world.

The potential impact of \gls{AP} on human health is amply documented.
Numerous papers have, for decades, established many links between air
quality and several common ailments like respiratory syndromes and
cardiovascular diseases. Similar connections have also been found
regarding the probability of gestational malformations and several types
of cancer. On a different level, and of perhaps less immediate concern,
are the effects that have been observed on ecosystems. Many times these
effects are difficult to predict (and timely mitigate) and in some cases
have been known to interfere with people's livelihood. In time, and if
not addressed, these interferences will certainly hinder economies and
limit the quality of life of populations globally. The severity of this
problem makes it clear that we need to tackle it intelligently, and this
approach requires that we can measure, trace and track \gls{AP}
effectively, which beckons engineers and scientists to create more
technology for this specific purpose.

Answering this call, with this work I have tried to create a reply to
the question of whether it would be possible to develop a
two-dimensional pollutant mapping tool, small enough to be fitted onto a
\gls{UAV}, which came to be a tomographically enabled design. To this
end, I have developed a simulation platform that computationally proves
the method's feasibility and confirmed through experiments that the
hypothesis on which the solution is based, regarding the use of
sequentially measured scattered sunlight as analogous to an artificial
light source is valid.

\subsection{Context}%
\label{sub:context}

The idea behind this thesis was born in 2015, at NGNS-IS (a Portuguese
tech startup). At the time, the company's flagship product was the
\gls{FFF}. The \gls{FFF} was a forest fire detection system, capable of
mostly autonomous and automatic operation.  The system was the first
application of \gls{DOAS} for fire detection, and for that it was
patented in 2007 (see~\cite{Vieira2007, Application2008}). The \gls{FFF}
is a remote sensing device that scans the horizon for the presence of a
smoke column, sequentially performing a chemical analysis of each
azimuth, using the Sun as a light source for its spectroscopic
operations~\cite{ValentedeAlmeida2017}.

The \gls{FFF} was deployed in several "habitats", both nationally
(Parque Nacional da Peneda-Gerês and Ourém) and internationally (Spain
and Brazil). One of the company's clients at the time was interested in
a pollution monitoring solution, and asked if the spectroscopic system
would be capable of performing such a task. The challenge resonated
through the company's structure and the idea that created this thesis
was born. The team then started reading about the concept of \gls{AP}
and how both populations and entities were concerned about it. It became
clear that, while there were already several methods to measure
\gls{AP}, there was a clear market drive for the development of a system
that could leverage the large area capabilities of a \gls{DOAS} device
while being able to provide a more spatially resolved "picture" of the
atmospheric status. With this in mind, the company managed to have the
investigation financed through a \gls{PT2020} funding opportunity. This
achievement was a clear validation of the project's goals and of the
need there was for a system with the proposed capabilities. It was,
however, not enough. \gls{FFF} was a very good starting point, but there
was still a lot of continuous research work needed before any of the
goals that had been set were achieved. This led to the publication of
this PhD project, in a tripartite consortium between FCT-NOVA, NGNS-IS
and the Portuguese Foundation for Science and Technology. Its main goal
was to develop an atmospheric monitoring system prototype that would be
able to spectroscopically map pollutant concentrations in a
two-dimensional way.

In April 2017, NGNS-IS was integrated in the Compta group, one of the
oldest IT groups operating in Portugal. Despite its age, this company is
one of the main presences in some of the most modern industrial fields,
like \gls{IOT} applications. \gls{ATMOS}'s pollutant tracing
capabilities made it an almost perfect fit in one of \gls{IOT}'s most
resounding niches, the \emph{Smart Cities} trend. Unfortunately, the
transition between one company and the other, regardless of the
project's adequacy, was anything but smooth. Almost two years later, in
the beginning of 2019, engulfed in a sea of endless bureaucracy and ill
intent on behalf of the managing governmental authorities (who seemed
always more interested in seeing the project fail than anything else),
\gls{ATMOS} was terminated and financing was cut.

\subsection{Research Questions}%
\label{sub:research_questions}

In Section~\ref{sub:context}, I have introduced the reasons which led
NGNS-IS to pursue the development of an atmospheric monitoring system,
and that what set it apart from other systems was the ability to
spectroscopically map pollutants concentrations using tomographic
methods, thus defining a primary objective for this thesis.

Two secondary objectives were born from the necessary initial research,
which had a very heavy influence over the adopted methods:
\begin{itemize}
    \item To use a tomographic approach for the mapping procedure;
    \item To ensure the designed system would be small and highly
        mobile;
    \item To use a single light collection point, minimizing material
        costs.
\end{itemize}

Taking all the above into account, we arrive at the main Research
Question (\gls{RQ}), presented in Table~\ref{tab:RQ1}.
\begin{table}[htpb]
    \centering
    \caption{Main research question.}
    \label{tab:RQ1}
    \begin{tabularx}{0.8\textwidth}{cX}
        \toprule
        \textbf{RQ1}&\emph{ How to design a miniaturized tomographic
        atmosphere monitoring system based on DOAS? }\\
        \bottomrule
    \end{tabularx}
\end{table}

This is the main research question. It gave rise to four other more
detailed research questions. These secondary questions allow a better
delimitation of the work at hand and are important complements to RQ1.
This questions are presented in Table~\ref{tab:sec_RQ}

\begin{table}[htpb]
    \centering
    \caption{Secondary research questions.}
    \label{tab:sec_RQ}
    \begin{tabularx}{0.8\textwidth}{cX}
        \toprule
        \textbf{RQ1.1}&\emph{ What would be the best strategy
        for the system to cover a small geographic region? }\\
        \midrule
        \textbf{RQ1.2}&\emph{ What would be the necessary
        components for such a system? }\\
        \midrule
        \textbf{RQ1.3}&\emph{ How will the system acquire the
        data? }\\
        \midrule
        \textbf{RQ1.4}&\emph{ What should the tomographic
        reconstruction look like and how to perform it? }\\
        \bottomrule
    \end{tabularx}
\end{table}

\section{Problem Introduction}%
\label{sec:problem_introduction}

Daniel Vallero, in his book "Fundamentals of Air
Pollution"~\cite{Vallero2014} makes a very important observation: Air
Pollution has no universal definition. Its meaning is intertwined with
the context with which it is measured and observed, with the ecosystem
in which it is perceived and even with the pollutant concentration (not
every toxic compound is toxic at every concentration). The \gls{EPA}
defines Air Pollution as the following:

\begin{center}
    \begin{minipage}{0.8\textwidth}

        \noindent
            \textit{Air Pollution is the presence of contaminants or pollutant
            substances in the air that interfere with human health or welfare,
            or produce other harmful environmental effects.}

    \end{minipage}
\end{center}

He then analyzes this definition through two possible lenses, the one
that comes with the interference produced by air contaminants; and the
one that comes from the harm they may cause. He notes that both points
of view come with a heavy burden of ambiguity, incompatible with a
scientific definition. We can thus observe that preferable to address
the issue through its measurable effects and consequences. These are
well-established and well known, and scientists all around the world
have been publishing extensively about them for some decades now. The
correlation between Air Pollution and an increased mortality in heavily
industrialized areas was first established in Europe, in the 19$^{th}$
century, but the first time it was taken seriously was during the 1952
killer-smog incidents, in London~\cite{Platt2007}. At the time, a
combination of very cold weather, an anticyclone and fireplace emissions
caused a thick smog to fall over London, directly causing thousands of
deaths and indirectly many more~\cite{Bell2008,Office2019}. The
disastrous consequences of this incident had a huge impact in the civil
society, resulting in a series of policies and laws, among which the
Clean Air Acts of 1956 and 1968.

\subsection{\acrlong{AP} Effects on Human Health}%
\label{sub:ap_effects_on_human_health}

\begin{flushright}
    \begin{minipage}{0.6\textwidth}
        ~\\[.5cm]
        \noindent
        \bfseries
        \textit{
            What is it that is not poison? All things are poison and
            nothing is without poison. It is the dose alone that makes a
            thing not poison.
        }

        \hfill-- Paracelsus
    \end{minipage}
\end{flushright}

This quote, originally in the writings of one of the fathers of modern
medicine, the Swiss Paracelsus, was taken from Patricia Frank's book
called \emph{The Dose Makes The Poison}~\cite{Frank2011} and is one of
the core tenets of toxicology even today. It holds for all chemicals,
including the ones that constitute \gls{AP}, but it is incomplete. It is
not the dose alone to make the poison. Rather, one has to consider the
whole context at which we are directing our discussion. Nitrogen
compounds in the air cause several respiratory syndromes, but in the
soil they are essential nutrients~\cite{Vallero2014, Lovett2009}.
Admittedly this is an extreme comparison, but it serves to introduce
the reader to the concept of dependence between the poisonous nature of
a chemical and the exposure conditions and circumstances. Moreover, it
is the precursor to another important notion, which is that of the
relation between dose and response. Generally, one can express the
adverse effect of a given chemical by the damage it causes in relation
to the exposure, the dose. Hence, increasing the concentration of a
chemical in an organism also increases the severity of the adverse
outcome~\cite{Vallero2014}. This mechanism, in which a variation of dose
implies a change in the organism's response is what is called the
dose-response, and is an important chemical characterization method in
this context.

\subsubsection{Respiratory System Ailments Related to \acrlong{AP}}%
\label{ssub:respiratory_system_ailments_related_to_ap}


\lipsum{5}
\begin{itemize}
    \item Cardiovascular
    \item Gestation and malformations
    \item Cancer
    \item Neurological
\end{itemize}


\begin{itemize}
    \item Ecosystem effects of Air pollution
    \item Air pollution and Climate change
    \item Sources of AP
    \item Air Pollution in Europe
    \item Air Pollution in other Western countries
    \item Air pollution in developing countries
    \item Detecting and monitoring air pollution
    \item Technology's role in tackling AP
\end{itemize}

% \section{Literature review}%
% \label{sec:literature_review}

% \subsection{Tomography}%
% \label{sub:tomography}

% \subsubsection{Reconstruction algorithms}%
% \label{ssub:reconstruction_algorithms}

% \begin{itemize}
%     \item Analytical
%         \begin{itemize}
%             \item FBP;
%             \item Fan-beam FBP.
%         \end{itemize}
%     \item Iterative
%         \begin{itemize}
%             \item SART;
%             \item SIRT;
%             \item MLEM;
%         \end{itemize}
% \end{itemize}

% \subsubsection{Geometries}%
% \label{ssub:geometries}

% \subsection{DOAS}%
% \label{sub:doas}

% \subsubsection{Background}%
% \label{ssub:background}

% \subsubsection{DOAS-tomography}%
% \label{ssub:doas_tomography}

% \section{Terms and scope of topics}%
% \label{sec:terms_and_scope_of_topics}

% \section{Knowledge gap}%
% \label{sec:knowledge_gap}

% \subsection{Smart air pollution monitoring}%
% \label{sub:smart_air_pollution_monitoring}

% \subsection{2d or 3d concentration mapping technologies}%
% \label{sub:2d_or_3d_concentration_mapping_technologies}

% \subsection{Low mobility of existing systems}%
% \label{sub:low_mobility_of_existing_systems}

% \section{Relevance of topic}%
% \label{sec:relevance_of_topic}

% \subsection{Smart cities}%
% \label{sub:smart_cities}

% \subsection{Air pollution monitoring in the future}%
% \label{sub:air_pollution_monitoring_in_the_future}

% \section{Research Questions}%
% \label{sec:research_questions}

% \section{Hypothesis}%
% \label{sec:hypothesis}

% \section{Methods}%
% \label{sec:methods}

% \subsection{Simulator}%
% \label{sub:simulator}

% \subsection{UAV Instruments}%
% \label{sub:uav_instruments}

% \subsection{Telescope and adaptations}%
% \label{sub:telescope_and_adaptations}

% \subsubsection{Model selection}%
% \label{ssub:model_selection}

% \subsubsection{Spectrometer coupling}%
% \label{ssub:spectrometer_coupling}

% \subsection{Experiments}%
% \label{sub:experiments}

% \subsection{SITL for trajectory programming}%
% \label{sub:sitl_for_trajectory_programming}

% \section{Findings}%
% \label{sec:findings}

% \section{Layout}%
% \label{sec:layout}














