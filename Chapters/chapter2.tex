%!TEX root = ../template.tex
%%%%%%%%%%%%%%%%%%%%%%%%%%%%%%%%%%%%%%%%%%%%%%%%%%%%%%%%%%%%%%%%%%%%
%% chapter2.tex
%% NOVA thesis document file
%%
%% Chapter with the template manual
%%%%%%%%%%%%%%%%%%%%%%%%%%%%%%%%%%%%%%%%%%%%%%%%%%%%%%%%%%%%%%%%%%%%
\chapter{Research Question}
\label{cha:research_question}

\section{Problem Introduction}%
\label{sec:problem_introduction}

\gls{AP} is a very important topic of discussion in the current days,
with scientists and researchers around the globe being very well aware
of the potential effects it can have on the health of individuals and
populations across all ecosystems. Not to mention its implications on
climate change, which are generally regarded as one of the capital
threats to life on Earth's survival (on par with a nuclear apocalypse).
Defining \gls{AP} can be a challenge. In fact, its effects and presence
is so all-encompassing, that it would be fair to say that its definition
changes with the angle with which one looks upon it.  Nonetheless, it is
important to at least try to define it, in order to approach it in some
way~\cite{Lovett2009, Ghorani-Azam2016}.

The United States Environmental Protection Agency (\Gls{EPA}) defined
Air Pollution (\Gls{AP}) as "\textit{the presence of contaminants or
    pollutant substances in the air that interfere with human health or
    welfare, or produce other harmful environmental
effects}"~\cite{Vallero2014}. This is (perhaps intentionally) a very
broad definition, too broad to avoid vagueness. It does introduce a key
concept: the term \emph{pollutant}, which needs be discussed in order
to complete the definition above.

It would be very hard to find someone who did not have an almost
instinctive idea of what a pollutant is. We know something is amiss when
we notice our air is full of smoke or smells strange, but our senses are
not enough. There are many chemical components that are untraceable by
unaided humans, and some that are only detected by our noses and eyes at
concentration levels which are above the threshold where they can damage
our health. This makes the task of separating pollutants from
non-pollutants a non-trivial one. If we cannot rely solely on our senses
to detect them, then it is up to the scientists and engineers to create
methods that allow us to do so. Whats more, we must also rely on them to
understand how can a normally harmless substance be a pollutant,
depending on the circumstance. For instance, nitrous compounds are
traditionally beneficial to the soils and cultures, but they can and do
cause pulmonary and  cardiovascular complications in
humans~\cite{Kampa2008, Ghorani-Azam2016, Carugno2016}.

Context matters to pollutants. The toxic nature of a certain chemical
only is revealed when someone or something gets exposed to it. Even
then, there are exposure levels which do not bear any effects, good or
bad. At these levels, a pollutant is but an impurity. There too many
potential pollutants in our modern day world to list here, but the World
Health Organization (\gls{WHO}) states that there are six major air
pollutants:
\begin{itemize}
    \item Particle Matter (\gls{PM});
    \item Ground level ozone (O$_3$);
    \item Carbon monoxide (CO);
    \item Sulfur Oxides (SO$_x$);
    \item Nitrous Oxides (NO$_x$);
    \item Lead (Pb).
\end{itemize}

Exposure to these pollutants have different effects on humans, ranging in
seriousness from skin irritation to neuropsychiatric complications,
depending on dose and on the time the exposure lasts.

\section{Research Question}%
\label{sec:research_question}

In Chapter~\ref{cha:introduction}, I have introduced the reasons which
led NGNS-IS to pursue the development of an atmospheric monitoring
system, and that what set it apart from other systems was the ability to
spectroscopically map pollutants concentrations using tomographic
methods, thus defining a primary objective for this thesis.

Two secondary objectives were born from the necessary initial research,
which had a very heavy influence over the adopted methods:
\begin{itemize}
    \item To use a tomographic approach for the mapping procedure;
    \item To use a single light collection point, minimizing material
        costs.
\end{itemize}

Taking all the above into account, we arrive at the main Research
Question (\gls{RQ}), presented in Table~\ref{tab:RQ1}.
\begin{table}[htpb]
    \centering
    \label{tab:RQ1}
    \caption{Main research question.}
    \begin{tabularx}{0.8\textwidth}{cX}
        \toprule
        \textbf{RQ1}&\emph{ How to design a tomographic atmosphere
        monitoring system based on DOAS? }\\
        \bottomrule
    \end{tabularx}
\end{table}

This is the main research question. It gave rise to four other more
detailed research questions. These secondary questions allow a better
delimitation of the work at hand and are important complements to RQ1.
This questions are presented in Table~\ref{tab:sec_RQ}

\begin{table}[htpb]
    \centering
    \label{tab:sec_RQ}
    \caption{Secondary research questions.}
    \begin{tabularx}{0.8\textwidth}{cX}
        \toprule
        \textbf{RQ1.1}&\emph{ What would be the best strategy
        for the system to cover a small geographic region in an urban
        setting }\\
        \midrule
        \textbf{RQ1.2}&\emph{ What would be the necessary
        components for such a system? }\\
        \midrule
        \textbf{RQ1.3}&\emph{ How will the system acquire the
        data? }\\
        \midrule
        \textbf{RQ1.4}&\emph{ What should the tomographic
        reconstruction look like and how to perform it? }\\
        \bottomrule
    \end{tabularx}
\end{table}



\section{Hypothesis and Approach}%
\label{sec:hypothesis_and_approach}


