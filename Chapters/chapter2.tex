%!TEX root = ../template.tex
%%%%%%%%%%%%%%%%%%%%%%%%%%%%%%%%%%%%%%%%%%%%%%%%%%%%%%%%%%%%%%%%%%%%
%% chapter2.tex
%% NOVA thesis document file
%%
%% Chapter with the template manual
%%%%%%%%%%%%%%%%%%%%%%%%%%%%%%%%%%%%%%%%%%%%%%%%%%%%%%%%%%%%%%%%%%%%
\chapter{Research Question}
\label{cha:research_question}

\section{Problem Introduction}%
\label{sec:problem_introduction}

\gls{AP} is a very important topic of discussion in the current days,
with scientists and researchers around the globe being very well aware
of the potential effects it can have on the health of individuals and
populations across all ecosystems. After climate change (one of the
largest capital threats to life on Earth, perhaps just behind nuclear
apocalypse), \gls{AP} is the biggest environmental concern for
Europeans, and Europe's single largest environmental health
hazard~\cite{EEA2016}. It has also been established by many authors as a
major cause of premature death, cardiopulmonary disease onset and
hospital visits~\cite{Ghorani-Azam2016, EEA2007, EEA2016,
WorldHealthOrganisationEurope2004,}.

Even in spite of all the trouble it causes, defining \gls{AP} can be a
challenge. In fact, its effects and presence is so all-encompassing,
that it would be fair to say that its definition changes with the angle
with which one looks upon it. The United States Environmental Protection
Agency (\Gls{EPA}) defined Air Pollution (\Gls{AP}) as "\textit{the
presence of contaminants or pollutant substances in the air that
interfere with human health or welfare, or produce other harmful
environmental effects}"~\cite{Vallero2014}. This is (perhaps
intentionally) a very broad definition, too broad to avoid vagueness. It
does introduce a key concept: the term \emph{pollutant}, which needs be
discussed in order to complete the definition above.

It would be very hard to find someone who did not have an almost
instinctive idea of what a pollutant is. We know something is amiss when
we notice our air is full of smoke or smells strange, but our senses are
not enough. There are many chemical components that are untraceable by
unaided humans, and some that are only detected by our noses and eyes at
concentration levels which are above the threshold where they can damage
our health. This makes the task of separating pollutants from
non-pollutants a non-trivial one. If we cannot rely solely on our senses
to detect them, then it is up to the scientists and engineers to create
methods that allow us to do so. Whats more, we must also rely on them to
understand how can a normally harmless substance be a pollutant,
depending on the circumstance. For instance, nitrous compounds are
traditionally beneficial to the soils and cultures, but they can and do
cause pulmonary and  cardiovascular complications in
humans~\cite{Kampa2008, Ghorani-Azam2016, Carugno2016}.

Context matters to pollutants. The toxic nature of a certain chemical
only is revealed when someone or something gets exposed to it. Even
then, there are exposure levels which do not bear any effects, good or
bad. At these levels, a pollutant is but an impurity. There are too many
potential pollutants in our modern day world to list here, but several
reports make special mention to six pollutions, which are identified as
being the major contributors to Air Pollution complications.

\begin{itemize}
    \item Particle Matter (\gls{PM});
    \item Ground level ozone (O$_3$);
    \item Carbon monoxide (CO);
    \item Sulfur Oxides (SO$_x$);
    \item Nitrous Oxides (NO$_x$);
    \item Lead (Pb).
\end{itemize}

Exposure to these pollutants has different effects on humans, ranging in
seriousness from skin irritation to neuropsychiatric complications,
depending on dose and on the time the exposure lasts.

\section{Research Question}%
\label{sec:research_question}

In Chapter~\ref{cha:introduction}, I have introduced the reasons which
led NGNS-IS to pursue the development of an atmospheric monitoring
system, and that what set it apart from other systems was the ability to
spectroscopically map pollutants concentrations using tomographic
methods, thus defining a primary objective for this thesis.

Two secondary objectives were born from the necessary initial research,
which had a very heavy influence over the adopted methods:
\begin{itemize}
    \item To use a tomographic approach for the mapping procedure;
    \item To ensure the designed system would be small and highly
        mobile;
    \item To use a single light collection point, minimizing material
        costs.
\end{itemize}

Taking all the above into account, we arrive at the main Research
Question (\gls{RQ}), presented in Table~\ref{tab:RQ1}.
\begin{table}[htpb]
    \centering
    \caption{Main research question.}
    \label{tab:RQ1}
    \begin{tabularx}{0.8\textwidth}{cX}
        \toprule
        \textbf{RQ1}&\emph{ How to design a miniaturized tomographic
        atmosphere monitoring system based on DOAS? }\\
        \bottomrule
    \end{tabularx}
\end{table}

This is the main research question. It gave rise to four other more
detailed research questions. These secondary questions allow a better
delimitation of the work at hand and are important complements to RQ1.
This questions are presented in Table~\ref{tab:sec_RQ}

\begin{table}[htpb]
    \centering
    \caption{Secondary research questions.}
    \label{tab:sec_RQ}
    \begin{tabularx}{0.8\textwidth}{cX}
        \toprule
        \textbf{RQ1.1}&\emph{ What would be the best strategy
        for the system to cover a small geographic region? }\\
        \midrule
        \textbf{RQ1.2}&\emph{ What would be the necessary
        components for such a system? }\\
        \midrule
        \textbf{RQ1.3}&\emph{ How will the system acquire the
        data? }\\
        \midrule
        \textbf{RQ1.4}&\emph{ What should the tomographic
        reconstruction look like and how to perform it? }\\
        \bottomrule
    \end{tabularx}
\end{table}



\section{Hypothesis and Approach}%
\label{sec:hypothesis_and_approach}

This work is based on the hypothesis that a system such as the one
described in Chapter~\ref{cha:introduction}, which responds to the
\gls{RQ} in Table~\ref{tab:RQ1} and Table~\ref{tab:sec_RQ} can be
achieved by careful selection of mathematical tomographic algorithms and
instrumentation that is able to implement them correctly.

The first step in answering the entirety of the research questions
should be to answer RQ1.1. In fact, it is not possible to make any other
decision before this matter is settled. As with any technical problem,
there are several ways to create a tomographic atmospheric monitoring
tool. However, each and every one of them implies some kind of
compromise, which determines the system's capabilities and requirements.
Will the system use retro-reflection? Shall it move during the
measurement? These are the kind of questions that determine the whole
project.

When the measurement strategy is determined, one could start picking
parts and components. However, a better first approach would be
designing a software simulation. This simulator must include all major
system features, so that it correctly mimics reality and is therefore
able to mathematically validate the acquisition and reconstruction
approach. The results obtained from the simulation will then dictate
mechanical and control requirements.

One other aspect that needs addressing is the optical section. As
mentioned before, the system will be inspired in \gls{FFF}'s basic
optical capabilities. However, the smoke detector was not conceived with
spatial restrictions in mind. This important set of components will thus
need redesigning, so that it is in line with the size objectives of the
new system.
