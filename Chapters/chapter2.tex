%!TEX root = ../template.tex
%%%%%%%%%%%%%%%%%%%%%%%%%%%%%%%%%%%%%%%%%%%%%%%%%%%%%%%%%%%%%%%%%%%%
%% chapter2.tex
%% NOVA thesis document file
%%
%% Chapter with the template manual
%%%%%%%%%%%%%%%%%%%%%%%%%%%%%%%%%%%%%%%%%%%%%%%%%%%%%%%%%%%%%%%%%%%%
\chapter{Research Question}
\label{cha:research_question}

\section{Problem Introduction}%
\label{sec:problem_introduction}


\gls{AP} is a very important topic of discussion in the current days,
with scientists and researchers around the globe being very well aware
of the potential effects it can have on the health of individuals and
populations across all ecosystems~\cite{Lovett2009}. After climate
change (one of the largest capital threats to life on Earth, perhaps
just behind nuclear apocalypse), \gls{AP} is the biggest environmental
concern for Europeans, and Europe's single largest environmental health
hazard~\cite{EEA2016}. It has also been established by many authors as a
major cause of premature death, cardiopulmonary disease onset and
hospital visits~\cite{Ghorani-Azam2016, EEA2007, EEA2016,
WorldHealthOrganisationEurope2004}.

The growing concerns about Air Pollution and its effects on human health
and the world in general is an indication for the importance of
measuring it correctly and with great detail. The diversity of its
effects and the sheer number of variables that need to be considered
establish the problem of \gls{AP} as one that can only be effectively
tackled if  approached intelligently, highlighting the need for smart
devices for the measuring and monitoring atmospheric pollutants.

At the moment, there are several solutions that can be implemented for
measuring atmospheric pollution. However, these solutions are limited in
their application to either large area coverage and small details (DOAS
ground stations, satellite data) or very detailed information in a very
localized and fixed way (in-situ electrochemical sensors). To the best
of our knowledge, there are no available systems that can monitor and
map pollutant concentration without requiring infrastructure
installation and capable of performing at various altitudes.

The answer to this lack could be a highly mobile tomographic DOAS
system, that could bridge the gap between the local monitoring
capabilities of in-situ electrochemical sensors and large-area
spectroscopic ground stations, while maintaining portability and
flexibility.

\section{Research Question}%
\label{sec:research_question}

In Chapter~\ref{cha:bg_and_motivation}, I have introduced the reasons
which led NGNS-IS to pursue the development of an atmospheric monitoring
system, and that what set it apart from other systems was the ability to
spectroscopically map pollutants concentrations using tomographic
methods, thus defining a primary objective for this thesis.

Two secondary objectives were born from the necessary initial research,
which had a very heavy influence over the adopted methods:
\begin{itemize}
    \item To use a tomographic approach for the mapping procedure;
    \item To ensure the designed system would be small and highly
        mobile;
    \item To use a single light collection point, minimizing material
        costs.
\end{itemize}

Taking all the above into account, we arrive at the main Research
Question (\gls{RQ}), presented in Table~\ref{tab:RQ1}.
\begin{table}[htpb]
    \centering
    \caption{Main research question.}
    \label{tab:RQ1}
    \begin{tabularx}{0.8\textwidth}{cX}
        \toprule
        \textbf{RQ1}&\emph{ How to design a miniaturized tomographic
        atmosphere monitoring system based on DOAS? }\\
        \bottomrule
    \end{tabularx}
\end{table}

This is the main research question. It gave rise to four other more
detailed research questions. These secondary questions allow a better
delimitation of the work at hand and are important complements to RQ1.
This questions are presented in Table~\ref{tab:sec_RQ}

\begin{table}[htpb]
    \centering
    \caption{Secondary research questions.}
    \label{tab:sec_RQ}
    \begin{tabularx}{0.8\textwidth}{cX}
        \toprule
        \textbf{RQ1.1}&\emph{ What would be the best strategy
        for the system to cover a small geographic region? }\\
        \midrule
        \textbf{RQ1.2}&\emph{ What would be the necessary
        components for such a system? }\\
        \midrule
        \textbf{RQ1.3}&\emph{ How will the system acquire the
        data? }\\
        \midrule
        \textbf{RQ1.4}&\emph{ What should the tomographic
        reconstruction look like and how to perform it? }\\
        \bottomrule
    \end{tabularx}
\end{table}



\section{Hypothesis and Approach}%
\label{sec:hypothesis_and_approach}

This work is based on the hypothesis that a system such as the one
described in Chapter~\ref{cha:bg_and_motivation}, which responds to the
\gls{RQ} in Table~\ref{tab:RQ1} and Table~\ref{tab:sec_RQ} can be
achieved by careful selection of mathematical tomographic algorithms and
instrumentation that is able to implement them correctly.

The first step in answering the entirety of the research questions
should be to answer RQ1.1. In fact, it is not possible to make any other
decision before this matter is settled. As with any technical problem,
there are several ways to create a tomographic atmospheric monitoring
tool. However, each and every one of them implies some kind of
compromise, which determines the system's capabilities and requirements.
Will the system use retro-reflection? Shall it move during the
measurement? These are the kind of questions that determine the whole
project.

When the measurement strategy is determined, one could start picking
parts and components. However, a better first approach would be
designing a software simulation. This simulator must include all major
system features, so that it correctly mimics reality and is therefore
able to mathematically validate the acquisition and reconstruction
approach. The results obtained from the simulation will then dictate
mechanical and control requirements.

One other aspect that needs addressing is the optical section. As
mentioned before, the system will be inspired in \gls{FFF}'s basic
optical capabilities. However, the smoke detector was not conceived with
spatial restrictions in mind. This important set of components will thus
need redesigning, so that it is in line with the size objectives of the
new system.
