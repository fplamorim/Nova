%!TEX root = ../template.tex
%%%%%%%%%%%%%%%%%%%%%%%%%%%%%%%%%%%%%%%%%%%%%%%%%%%%%%%%%%%%%%%%%%%%
%% abstrac-pt.tex
%% NOVA thesis document file
%%
%% Abstract in Portuguese
%%%%%%%%%%%%%%%%%%%%%%%%%%%%%%%%%%%%%%%%%%%%%%%%%%%%%%%%%%%%%%%%%%%%

\typeout{NT FILE dg_abstract-pt.tex}%

Os animais adaptam as suas decisões de procura de alimentos com respeito à disponibilidade de nutrientes e às suas necessidades internas. Estas decisões dependem de cálculos complexos a fim de decidir se exploram uma fonte alimentar ou se continuam a explorar opções potencialmente melhores. No entanto, a lógica do circuito neuronal subjacente aos cálculos dinâmicos para equilibrar a exploração e exploração durante a forrageira e a forma como são moldados pelos estados internos permanecem mal compreendidos.

\indent Nesta tese, apresento uma abordagem metodológica para estudar como a mosca da fruta comum \textit{Drosophila melanogaster} decide onde e o que comer a fim de satisfazer as suas necessidades nutricionais actuais e potenciais futuras. Este método combina a quantificação comportamental detalhada e automatizada de alto rendimento com um ecrã de manipulação neurogenética em grande escala para identificar populações neuronais implicadas em diferentes aspectos da tomada de decisões das moscas durante a procura de alimentos. Especificamente, criámos uma configuração chamada \textit{optoPAD} como um sistema optogenético de ciclo fechado para estudar os circuitos neuronais envolvidos no comportamento alimentar. Além disso, estendendo o paradigma da escolha alimentar a uma arena de forragens com múltiplas manchas de levedura e sacarose, estudamos as decisões de forragens tomadas pelas moscas enquanto exploram a arena. Usando esta configuração, realizámos um ecrã de comportamento neurogenético em grande escala para testar o efeito da actividade de silenciamento em diferentes subconjuntos neuronais em aspectos de forragens. Descobrimos uma população neuronal que se projecta para uma região central do cérebro do insecto, o que é importante para a decisão de parar numa mancha alimentar. Mostrámos que um único par de neurónios dentro desta população é necessário para o impulso exploratório da mosca, e que a actividade neste neurónio é modulada pelo estado proteico da mosca. Finalmente, apresento conclusões deste estudo a fim de formar um quadro teórico integrador da tomada de decisões das moscas durante a forragem, com o objectivo normativo de equilibrar a ingestão de nutrientes.

\indent Os nossos resultados revelam um substrato neural importante para a regulação de decisões complexas e etnologicamente relevantes em matéria de forragens. Isto fornece um passo fundamental para uma explicação mecanicista das funções cognitivas necessárias para calcular os trade-offs de exploração-exploração, e como tal, como combinar os resultados da procura de alimentos com as necessidades fisiológicas.

\keywords{
  Neurociência de Sistemas \and
  Comportamento \and
  Procura de alimentos \and
  Exploração \and
  Computação Neuronal\and
  Cognição
}
% to add an extra black line
