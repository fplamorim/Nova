%!TEX root = ../template.tex
%%%%%%%%%%%%%%%%%%%%%%%%%%%%%%%%%%%%%%%%%%%%%%%%%%%%%%%%%%%%%%%%%%%%
%% abstrac-en.tex
%% NOVA thesis document file
%%
%% Abstract in English([^%]*)
%%%%%%%%%%%%%%%%%%%%%%%%%%%%%%%%%%%%%%%%%%%%%%%%%%%%%%%%%%%%%%%%%%%%

\typeout{NT FILE dg_abstract-en.tex}%

Animals adapt their foraging decisions with respect to nutrient availability and their internal needs. These decisions rely on complex computations in order to decide whether to exploit a food source or to continue exploring potentially better options. However, the neuronal circuit logic underlying dynamic computations for balancing exploration and exploitation during foraging and how they are shaped by internal states remain poorly understood.

\indent In this thesis, I present a methodological approach to studying how the common fruit fly \textit{Drosophila melanogaster} decides where and what to eat in order to fulfil their current and potential future nutrient needs. This method combines detailed and high-throughput, automated behavioral quantification with a large-scale neurogenetic manipulation screen to identify neuronal populations implicated in different aspects of the flies' decision-making during foraging. Specifically, we created a setup called \textit{optoPAD} as a closed-loop optogenetics system to study neuronal circuits involved in feeding behavior. Furthermore, extending the food choice paradigm to a foraging arena with multiple yeast and sucrose patches, we study foraging decisions made by flies as they explore the arena. Using this setup, we carried out a large-scale neurogenetic behavioral screen to test the effect of silencing activity in different neuronal subsets on aspects of foraging. We discovered a neuronal population that projects to a central region of the insect brain, which is important for the decision to stop at a food patch. We showed that a single pair of neurons within this population is required for the fly's exploratory drive, and that the activity in this neuron is modulated by the fly's protein state. Finally, I present conclusions of this study in order to form an integrative theoretical framework of the flies' decision-making during foraging with the normative goal to balance nutrient intake.

\indent Our results reveal a neural substrate important for regulating complex, ethologically relevant foraging decisions. This provides a key step towards a mechanistic explanation of cognitive functions required to compute the exploration-exploitation trade-offs, and as such how to match foraging outcomes to physiological needs.
\keywords{
  Systems Neuroscience \and
  Behavior \and
  Foraging \and
  Exploration \and
  Neuronal Computation\and
  Cognition
}
%Regardless of the language in which the dissertation is written, usually there are at least two abstracts: one abstract in the same language as the main text, and another abstract in some other language.
%
%The abstracts' order varies with the school.  The default behaviour for the \gls{novathesis} template is to have in first place the abstract in \emph{the same language as main text}, and then the abstract in \emph{the other language}. For example, if the dissertation is written in Portuguese, the abstract order will be first Portuguese and then English, followed by the main text in Portuguese. If the dissertation is written in English, the abstract order will be first English and then Portuguese, followed by the main text in English.
%%
%The \gls{novathesis} (\LaTeX) template will automatically order the abstracts by considering this rule. However, this order can be customized by adding
%\begin{verbatim}
%    \abstractorder(<MAIN_LANG>):={<LANG_1>,...,<LANG_N>}
%\end{verbatim}
%\noindent to the file \verb!5_packages.tex!.  For example, for a main document written in German with abstracts written in German, English and Italian (by this order) use:
%\begin{verbatim}
%    \abstractorder(de):={de,en,it}
%\end{verbatim}
%
%Concerning its contents, the abstracts should not exceed one page and may answer the following questions (it is essential to adapt to the usual practices of your scientific area):
%
%\begin{enumerate}
%  \item What is the problem?
%  \item Why is this problem interesting/challenging?
%  \item What is the proposed approach/solution/contribution?
%  \item What results (implications/consequences) from the solution?
%\end{enumerate}

% Palavras-chave do resumo em Inglês
% \begin{keywords}
% Keyword 1, Keyword 2, Keyword 3, Keyword 4, Keyword 5, Keyword 6, Keyword 7, Keyword 8, Keyword 9
% \end{keywords}
