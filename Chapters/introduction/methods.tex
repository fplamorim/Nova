%!TEX root = ../../thesis_rui_almeida.tex


\section{Methods and Findings}%
\label{sec:intro_methods}

\todo[inline]{Consider rewriting this section.}

The system proposed in this thesis is a conjunction of sub-components
that work together towards the final goal. The first division is between
what is a physical, tangible component and what is just software
related. On the first level, developing this project involved selecting
the components for a custom made \gls{UAV} to which the spectroscopic
system will be attached to. In addition to this, I also had to select
the on-board electronics module, which is comprised of the flight
controller, which takes care of the device's movement; and the on-board
computer, which handles the acquisition of data via the optical module,
an assembly of spectrometers and telescopes connected through a
custom-built coupling part. All this assembly will perform according to
what is programmed using ArduCopter's SITL programming
suite~\todo{citation}.

On a software level, developing this project implied the creation of a
tomographic simulation tool, written in Python (and more particularly,
NumPy)~\cite{Python, Oliphant2007}. This tool was coded considering the
hypothesis described in Section~\ref{sec:intro_hypothesis}, and includes the
description of the trajectory that the system must conduct in order to
reconstruct the target species concentration maps. Regarding the
reconstruction itself, this tool includes custom-built routines for the
\gls{FBP} algorithm and the \gls{MLEM} algorithm, and applies a SciPy
library~\todo{citation} method to perform \gls{SART} reconstruction.
Moreover, the simulation platform also takes into consideration
geometric and reconstruction errors, proper to this kind of system.

The rest of this thesis describes my best efforts in developing a
spectral system that would provide an affirmative reply to the research
question, taking into account the literary gaps that we have found in
our literature review. In doing this, I have successfully built a
software simulation tool that computationally proves that our
tomographic assumptions are reasonable and allow the concentration
mapping of any trace species that can be targeted by the \gls{DOAS}
technique. On an experimental level, I have also managed to validate the
hypothesis described in Section~\ref{sec:intro_hypothesis}.
