%!TEX root = ../../thesis_rui_almeida.tex


\section{Hypothesis}%
\label{sec:intro_hypothesis}

As stated in Section~\ref{sec:research_questions}, the main goal of my
work was to provide an answer to the question of how to design a
miniaturized tomographic atmosphere monitoring system, based on
\gls{DOAS}.  This sentence is the most important point of the project.
Every development from this point forward stems from it and is motivated
by it. It also requires some deconstruction in order to understand the
true scope of the matter. First, it is a miniaturized device. This means
that besides all the habitual requirements (performance, function,
adequacy, compatibility with the other components, safety and security)
in defining components in an engineering project, one must also keep in
mind the footprint of each component, and how much it weighs. Second, it
is a tomographic system, which means that not only the device must be
able to take line integrals from some kind of medium, it must also be
able to describe a predefined trajectory, with admissible levels of
geometric error, which complies to a certain projection geometry.
Otherwise, one would not be able to apply a tomographic reconstruction
routine to obtain the map of the target species concentrations. Finally
the system is supposed to monitor the atmosphere in some way. Now, as
implied in the same sentence, \gls{DOAS} is the technique that I am
trying to apply in this system. But as I point out in
Section~\ref{sub:doas} and with more depth in
Section~\ref{sec:lit_review_doas} , there are two families of
\gls{DOAS}, and within them, many sub-techniques. This system which I am
developing is based on the hypothesis that we can use an almost hybrid
approach to \gls{DOAS}: passive \gls{DOAS} instrumental simplicity and
active \gls{DOAS} retrieval simplicity. As illustrated in
Figure~\ref{fig:hypothesis}, one could use a scattered sunlight
measurement as a light source for a \gls{DOAS} analysis, provided
distances between the two points are kept small, optical densities are
low (clear atmosphere), and both spectral measurements are taken in the
same angle. With this process, we would only effectively be using as
projections the spectral measurements of the \gls{ROI}. This hypothesis,
while not being mentioned in the library directly, has already been
hinted at in several papers, namely references~\cite{Frins2006,
Casaballe2017, Johansson2009}.

\begin{figure}[htpb]
    \centering
    \missingfigure{}
    \caption{Hypothesis schematic. Light captured at point A is used as
        a light source (or $I_0$) for the light captured at point B,
        just as if using an artificial light source. As long as
        distances are kept small and the optical thickness is low,
        scattering will be negligible.  This is a huge simplification
        over the traditional passive \gls{DOAS} analysis process, as
        explained in Section~\ref{sec:lit_review_doas}.}
    \label{fig:hypothesis}
\end{figure}
