%!TEX root = ../../thesis_rui_almeida.tex


\section{Problem Introduction}%
\label{sec:problem_introduction}

\acrlong{AP} poses an important threat to the human way of life. The
\gls{WHO} have estimated that 1 out of each 9 deaths in 2012 were
\gls{AP}-related and of these, 3 million were directly attributable to
outdoor \gls{AP} worldwide, most of which in developing countries (87\%
vs 82\% population). Although the European picture is not so dire as
this, the topic does cause concern. In 2016, there were an estimated
400.000 deaths due to \gls{AP} in Europe, 391.000 of which in the EU-28
space~\cite{Guerreiro2019}. An increased number of premature deaths is
sufficiently bad for treating this issue seriously, but the problems
brought forth by \acrlong{AP} do not end here. Not only are people dying
more, disabilities (namely respiratory) are more frequent, and so are
hospital visits. These two factors represent a decrease in productivity
and an increase in medical costs, which accrue to the huge burden that
\gls{AP} already represents to any society. In Europe, health impacts of
diesel emissions were estimated to be in the region of 60 billion
euros~\cite{CEDelft2018} for the year of 2016.

These impressive numbers have perspired onto the public opinion, which
is (now more than ever) concerned with the whole problem of \gls{AP}. In
fact, the subject is the considered by the public the second most
important environmental threat (after Climate Change, which is a very
related topic), and citizens throughout Europe have been partaking in
initiatives which aim to aid and incentivize air quality monitoring, as
well as raising awareness to the necessity of paying attention to this
issue and for behavioral changes. As tackling \acrlong{AP} and its
causes grows ever more popular, so does the political weight associated
with the subject, which in turn results in an increased number of
measures destined improve air quality. However, effective actions
against \gls{AP} require the approach to be intelligent and
knowledgeable, for the more we know, the better we can handle it. It is
thus the role of technology and technologists, to develop new ways in
which to measure, map, track and trace \gls{AP}, leveraging the power of
human intellect and ingenuity to combat this impending threat that is
upon us.
