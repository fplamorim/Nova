%!TEX root = ../thesis_rui_almeida.tex
%%%%%%%%%%%%%%%%%%%%%%%%%%%%%%%%%%%%%%%%%%%%%%%%%%%%%%%%%%%%%%%%%%%%
%% 2_litreview.tex
%% Rui V. Almeida's thesis file
%%
%% This chapter contains the literature review.
%%%%%%%%%%%%%%%%%%%%%%%%%%%%%%%%%%%%%%%%%%%%%%%%%%%%%%%%%%%%%%%%%%%%
\chapter{Literature Review}%
\label{cha:literature_review}

\section{\acrlong{AP}}%
\label{sec:ap}

Daniel Vallero, in his book "Fundamentals of Air
Pollution"~\cite{Vallero2014} makes a very important observation: Air
Pollution has no universal definition. Its meaning is intertwined with
the context with which it is measured and observed, with the ecosystem
in which it is perceived and even with the pollutant concentration (not
every toxic compound is toxic at every concentration). The \gls{EPA}
defines Air Pollution as the following:

\begin{center}
    \begin{minipage}{0.8\textwidth}

        \noindent
            \textit{Air Pollution is the presence of contaminants or pollutant
            substances in the air that interfere with human health or welfare,
            or produce other harmful environmental effects.}

    \end{minipage}
\end{center}

He then analyzes this definition through two possible lenses, the one
that comes with the interference produced by air contaminants; and the
one that comes from the harm they may cause. He notes that both points
of view come with a heavy burden of ambiguity, incompatible with a
scientific definition. We can thus observe that preferable to address
the issue through its measurable effects and consequences. These are
well-established and well known, and scientists all around the world
have been publishing extensively about them for some decades now. The
correlation between Air Pollution and an increased mortality in heavily
industrialized areas was first established in Europe, in the 19$^{th}$
century, but the first time it was taken seriously was during the 1952
killer-smog incidents, in London~\cite{Platt2007}. At the time, a
combination of very cold weather, an anticyclone and fireplace emissions
caused a thick smog to fall over London, directly causing thousands of
deaths and indirectly many more~\cite{Bell2008,Office2019}. The
disastrous consequences of this incident had a huge impact in the civil
society, resulting in a series of policies and laws, among which the
Clean Air Acts of 1956 and 1968.

\subsection{\acrlong{AP} Effects on Human Health}%
\label{sub:ap_effects_on_human_health}

\begin{flushright}
    \begin{minipage}{0.6\textwidth}
        ~\\[.5cm]
        \noindent
        \bfseries
        \textit{
            What is it that is not poison? All things are poison and
            nothing is without poison. It is the dose alone that makes a
            thing not poison.
        }

        \hfill-- Paracelsus
    \end{minipage}
\end{flushright}

This quote, originally in the writings of one of the fathers of modern
medicine, the Swiss Paracelsus, was taken from Patricia Frank's book
called \emph{The Dose Makes The Poison}~\cite{Frank2011} and is one of
the core tenets of toxicology even today. It holds for all chemicals
that constitute what we are used to think as \gls{AP}, but for all the
others as well. What I mean by this is an extension of what I have
previously introduced as a definition of pollution by its effects rather
than by its physical presence. The following section introduces how
humans can become affected by several chemicals which are usually part
of a developed country's atmosphere.


%%% THIS PARAGRAPH IS WELL WRITTEN, BUT DID NOT FIT THAT WELL
%%%%%%%%%%%%%%%%%%%%%%%%%%%

% It is not the dose alone to make the poison. Rather, one has to consider
% the whole context at which we are directing our discussion. Nitrogen
% compounds in the air cause several respiratory syndromes, but in the
% soil they are essential nutrients~\cite{Vallero2014, Lovett2009}.
% Admittedly this is an extreme comparison, but it serves to introduce the
% reader to the concept of dependence between the poisonous nature of a
% chemical and the exposure conditions and circumstances. Moreover, it is
% the precursor to another important notion, which is that of the relation
% between dose and response. Generally, one can express the adverse effect
% of a given chemical by the damage it causes in relation to the exposure,
% the dose. Hence, increasing the concentration of a chemical in an
% organism also increases the severity of the adverse
% outcome~\cite{Vallero2014}. This mechanism, in which a variation of dose
% implies a change in the organism's response is what is called the
% dose-response, and is an important chemical characterization method in
% this context.

\subsubsection{Respiratory System Ailments Related to \acrlong{AP}}%
\label{ssub:respiratory_system_ailments_related_to_ap}


% \lipsum{5}
\begin{itemize}
    \item Cardiovascular
    \item Gestation and malformations
    \item Cancer
    \item Neurological
\end{itemize}


\begin{itemize}
    \item Ecosystem effects of Air pollution
    \item Air pollution and Climate change
    \item Sources of AP
    \item Air Pollution in Europe
    \item Air Pollution in other Western countries
    \item Air pollution in developing countries
    \item Detecting and monitoring air pollution
    \item Technology's role in tackling AP
\end{itemize}

\section{DOAS}%
\label{sec:doas}

\section{Tomography}%
\label{sec:tomography}

\section{DOAS Tomography}%
\label{sec:doas_tomography}






