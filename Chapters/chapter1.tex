%!TEX root = ../template.tex
%%%%%%%%%%%%%%%%%%%%%%%%%%%%%%%%%%%%%%%%%%%%%%%%%%%%%%%%%%%%%%%%%%%
%% chapter1.tex
%% NOVA thesis document file
%%
%% Chapter with introduciton
%%%%%%%%%%%%%%%%%%%%%%%%%%%%%%%%%%%%%%%%%%%%%%%%%%%%%%%%%%%%%%%%%%%
\newcommand{\novathesis}{\emph{novathesis}}
\newcommand{\novathesisclass}{\texttt{novathesis.cls}}


\chapter{Introduction}
\label{cha:introduction}

\section{Conceptualization}%
\label{sec:contextualisation}

\subsection{Background}%
\label{sub:background}

\subsection{The Problem}%
\label{sub:the_problem}

\subsection{Objectives}%
\label{sub:objectives}

\subsection{Methods}%
\label{sub:methods}

\subsection{Results}%
\label{sub:results}

\subsection{Conclusions}%
\label{sub:conclusions}

\section{Theoretical Background}%
\label{sec:theoretical_background}

\subsection{Air Pollution}%
\label{ssub:air_pollution}

Air Pollution (\gls{AP}) is a very important topic of discussion in the
current days, with scientists and researchers around the globe being
very well aware of the potential effects it can have on the health of
individuals and populations across all ecosystems. Not to mention its
implications on climate change, which are generally regarded as one of
the capital threats to life on Earth's survival (on par with a nuclear
apocalypse). Defining \gls{AP} can be a challenge. In fact, its effects
and presence is so all-encompassing, that it would be fair to say that
its definition changes with the angle with which one looks upon it.
Nonetheless, it is important to at least try to define it, in order to
approach it in some way~\cite{Lovett2009, Ghorani-Azam2016}.

The United States Environmental Protection Agency (\Gls{EPA}) defined
Air Pollution (\Gls{AP}) as "\textit{the presence of contaminants or
    pollutant substances in the air that interfere with human health or
    welfare, or produce other harmful environmental
effects}"~\cite{Vallero2014}. This is (perhaps intentionally) a very
broad definition, too broad to avoid vagueness. It does introduce a key
concept: the term \emph{pollutant}, which needs be discussed in order
to complete the definition above.

It would be very hard to find someone who did not have an almost
instinctive idea of what a pollutant is. We know something is amiss when
we notice our air is full of smoke or smells strange, but our senses are
not enough. There are many chemical components that are untraceable by
unaided humans, and some that are only detected by our noses and eyes at
concentration levels which are above the threshold where they can damage
our health. This makes the task of separating pollutants from
non-pollutants a non-trivial one. If we cannot rely solely on our senses
to detect them, then it is up to the scientists and engineers to create
methods that allow us to do so. Whats more, we must also rely on them to
understand how can a normally harmless substance be a pollutant,
depending on the circumstance. For instance, nitrous compounds are
traditionally beneficial to the soils and cultures, but they can and do
cause pulmonary and  cardiovascular complications in
humans~\cite{Kampa2008, Ghorani-Azam2016, Carugno2016}.

Context matters to pollutants. The toxic nature of a certain chemical
only is revealed when someone or something gets exposed to it. Even
then, there are exposure levels which do not bear any effects, good or
bad. At these levels, a pollutant is but an impurity. There too many
potential pollutants in our modern day world to list here, but the World
Health Organization (\gls{WHO}) states that there are six major air
pollutants:
\begin{itemize}
    \item Particle Matter (\gls{PM});
    \item Ground level ozone (O$_3$);
    \item Carbon monoxide (CO);
    \item Sulfur Oxides (SO$_x$);
    \item Nitrous Oxides (NO$_x$);
    \item Lead (Pb).
\end{itemize}

Exposure to these pollutants have different effects on humans, ranging in
seriousness from skin irritation to neuropsychiatric complications,
depending on dose and on the time the exposure lasts. In the next
subsections, I will address each of the major pollutants enumerated and
briefly describe the physiological mechanisms behind their toxicity.

\subsubsection{Particle Matter}%
\label{ssub:particle_matter}

\subsubsection{Ground Level Ozone}%
\label{ssub:ground_level_ozone}

\subsubsection{Carbon Monoxide}%
\label{ssub:carbon_monoxide}

\subsubsection{Sulfur Oxides}%
\label{ssub:sulfur_oxides}

\subsubsection{Nitrous Oxides}%
\label{ssub:nitrous_oxides}

\subsubsection{Lead}%
\label{ssub:lead}
















\section{Literature Review}%
\label{sec:literature_review}
