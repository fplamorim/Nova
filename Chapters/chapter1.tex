%!TEX root = ../template.tex
%%%%%%%%%%%%%%%%%%%%%%%%%%%%%%%%%%%%%%%%%%%%%%%%%%%%%%%%%%%%%%%%%%%
%% chapter1.tex
%% NOVA thesis document file
%%
%% Chapter with introduciton
%%%%%%%%%%%%%%%%%%%%%%%%%%%%%%%%%%%%%%%%%%%%%%%%%%%%%%%%%%%%%%%%%%%
\newcommand{\novathesis}{\emph{novathesis}}
\newcommand{\novathesisclass}{\texttt{novathesis.cls}}


\chapter{Introduction}
\label{cha:introduction}

\section{Background and Motivation}%
\label{sec:bg_and_motivation}

\subsection{Context}%
\label{sub:context}

The idea behind this thesis was born in 2015, at NGNS-IS (a Portuguese
tech startup). At the time, the company's flagship product was the
Forest Fire Finder (\gls{FFF}). The \gls{FFF} was a forest fire
detection system, capable of mostly autonomous and automatic operation.
The system was the first application of Differential Optical Absorption
Spectroscopy~\gls{DOAS} for fire detection, and for that it was patented
in 2007 (see~\cite{Vieira2007, Application2008}). The \gls{FFF} is a
remote sensing device that scans the horizon for the presence of a smoke
column, sequentially performing a chemical analysis of each azimuth,
using the Sun as a light source for its spectroscopic
operations~\cite{ValentedeAlmeida2017}.

The \gls{FFF} was deployed in several "habitats", both nationally
(Parque Nacional da Peneda-Gerês and Ourém) and internationally (Spain
and Brazil). One of the company's clients at the time was interested in
a pollution monitoring solution, and asked if the spectroscopic system
would be capable of performing such a task. The challenge resonated
through the company's structure and the idea that created this thesis
was born. The team then started reading about the concept of Air
Pollution~\gls{AP} and how both populations and entities were concerned
about it. It became clear that, while there were already several methods
to measure \gls{AP}, there was a clear market drive for the development
of a system that could leverage the large area capabilities of a
\gls{DOAS} device while being able to provide a more spatially resolved
"picture" of the atmospheric status. With this in mind, the company
managed to have the investigation financed through a PT2020 funding
opportunity. This achievement was a clear validation of the project's
goals and of the need there was for a system with the proposed
capabilities. It was, however, not enough. \gls{FFF} was a very good
starting point, but there was still a lot of continuous research work
needed before any of the goals that had been set were achieved. This led
to the publication of this PhD project, in a tripartite consortium
between FCT-NOVA, NGNS-IS and the Portuguese Foundation for Science and
Technology. Its main goal was to develop an atmospheric monitoring
system prototype that would be able to spectroscopically map pollutant
concentrations in a two-dimensional way.

\subsection{The Problem}%
\label{sub:the_problem}

Air Pollution (\gls{AP}) is one of the grave concerns of modern day
western society, with many decades worth of research proving that it can
have a pronounced negative effect on human, animal and plant life, as
shown in Section~\ref{sec:theoretical_background}. On humans, it has
been shown to significantly increase risk of cardiovascular, pulmonary
and even neuropsychiatric diseases~\cite{Carugno2016, Ghorani-Azam2016,
Kampa2008}. Its implications on ecosystems are remarkably complex and
difficult to quantify, but nonetheless extremely important, and have a
huge impact on biodiversity~\cite{Lovett2009}.

Knowing all this brings us the responsibility of at least trying to
mitigate some of these adverse consequences of the spectacular progress
that we have achieved in the few last centuries. But we cannot act
unless we also know what we must do; and to know this, we must have
the devices and means to employ them that allow us to measure \gls{AP},
to trace it and to understand it, so that our actions against it are
effective.

\subsection{Objectives}%
\label{sub:objectives}

The overarching goal of this thesis was to theorize and design a
bidimensional mapping tool for trace atmospheric pollutants such as
NO$_x$ and SO$_x$, using DOAS as the measurement technique. In order to
maximize commercial value (and viability), the system had to be small
and mobile.During the research, several "micro-objectives" appeared
regularly. Some were kept and incorporated in the workplan, others
discarded after initial exploration. The main secondary objectives were:

\begin{itemize}
    \item To use a tomographic approach for the mapping procedure;
    \item To ensure the designed system would be small and highly
        mobile;
    \item To use a single collection point, minimizing material costs
        for the technology.

\end{itemize}

These objectives allowed setting several research questions, which are
introduced in Section~\ref{sec:research_question}.

\subsection{Methods}%
\label{sub:methods}

To address these goals, I assumed the development of this thesis to be
essentially split in two parts, which are to be explored simultaneously.
They can (coarsely) be addressed as \emph{tomography} and
\emph{instrumentation}. On the tomography side, it will be necessary to
study what are the more appropriate algorithms (and what type of
tomography), how they can be physically deployed (i.e., the problem's
geometry) and what type of reconstruction method is the more favorable.
Regarding the instrumentation, there are also several points that need
considering: decisions are required with regard to the mechanics, the
controls and the optical components of the final system. A more detailed
discussion of these topics can be found in
Section~\ref{sec:hypothesis_and_approach} and
Chapter~\ref{cha:research_methods}.

One of the most important steps in the development of this work is
designing and implementing a simulation software platform, that allows
the validation of the acquisition strategy, the geometry selection and
the reconstruction approach. This endeavor will also be of critical
importance for component selection, since it will define the component
requirements for the whole system. On the optical instrumentation side,
this project will require optimizing \gls{FFF}'s optical assembly.
Although similar in purpose and types of components, this assembly is
significantly larger than what is acceptable for this project and needs
redesigning.

% \section{Theoretical Background}%
% \label{sec:theoretical_background}

% \subsection{Air Pollution}%
% \label{ssub:air_pollution}

% Air Pollution (\gls{AP}) is a very important topic of discussion in the
% current days, with scientists and researchers around the globe being
% very well aware of the potential effects it can have on the health of
% individuals and populations across all ecosystems. Not to mention its
% implications on climate change, which are generally regarded as one of
% the capital threats to life on Earth's survival (on par with a nuclear
% apocalypse). Defining \gls{AP} can be a challenge. In fact, its effects
% and presence is so all-encompassing, that it would be fair to say that
% its definition changes with the angle with which one looks upon it.
% Nonetheless, it is important to at least try to define it, in order to
% approach it in some way~\cite{Lovett2009, Ghorani-Azam2016}.

% The United States Environmental Protection Agency (\Gls{EPA}) defined
% Air Pollution (\Gls{AP}) as "\textit{the presence of contaminants or
%     pollutant substances in the air that interfere with human health or
%     welfare, or produce other harmful environmental
% effects}"~\cite{Vallero2014}. This is (perhaps intentionally) a very
% broad definition, too broad to avoid vagueness. It does introduce a key
% concept: the term \emph{pollutant}, which needs be discussed in order
% to complete the definition above.

% It would be very hard to find someone who did not have an almost
% instinctive idea of what a pollutant is. We know something is amiss when
% we notice our air is full of smoke or smells strange, but our senses are
% not enough. There are many chemical components that are untraceable by
% unaided humans, and some that are only detected by our noses and eyes at
% concentration levels which are above the threshold where they can damage
% our health. This makes the task of separating pollutants from
% non-pollutants a non-trivial one. If we cannot rely solely on our senses
% to detect them, then it is up to the scientists and engineers to create
% methods that allow us to do so. Whats more, we must also rely on them to
% understand how can a normally harmless substance be a pollutant,
% depending on the circumstance. For instance, nitrous compounds are
% traditionally beneficial to the soils and cultures, but they can and do
% cause pulmonary and  cardiovascular complications in
% humans~\cite{Kampa2008, Ghorani-Azam2016, Carugno2016}.

% Context matters to pollutants. The toxic nature of a certain chemical
% only is revealed when someone or something gets exposed to it. Even
% then, there are exposure levels which do not bear any effects, good or
% bad. At these levels, a pollutant is but an impurity. There too many
% potential pollutants in our modern day world to list here, but the World
% Health Organization (\gls{WHO}) states that there are six major air
% pollutants:
% \begin{itemize}
%     \item Particle Matter (\gls{PM});
%     \item Ground level ozone (O$_3$);
%     \item Carbon monoxide (CO);
%     \item Sulfur Oxides (SO$_x$);
%     \item Nitrous Oxides (NO$_x$);
%     \item Lead (Pb).
% \end{itemize}

% Exposure to these pollutants have different effects on humans, ranging in
% seriousness from skin irritation to neuropsychiatric complications,
% depending on dose and on the time the exposure lasts. In the next
% subsections, I will address each of the major pollutants enumerated and
% briefly describe the physiological mechanisms behind their toxicity.

% \subsubsection{Particle Matter}%
% \label{ssub:particle_matter}

% \subsubsection{Ground Level Ozone}%
% \label{ssub:ground_level_ozone}

% \subsubsection{Carbon Monoxide}%
% \label{ssub:carbon_monoxide}

% \subsubsection{Sulfur Oxides}%
% \label{ssub:sulfur_oxides}

% \subsubsection{Nitrous Oxides}%
% \label{ssub:nitrous_oxides}

% \subsubsection{Lead}%
% \label{ssub:lead}
















% \section{Literature Review}%
% \label{sec:literature_review}
