%!TEX root = ../template.tex
%%%%%%%%%%%%%%%%%%%%%%%%%%%%%%%%%%%%%%%%%%%%%%%%%%%%%%%%%%%%%%%%%%%
%% chapter1.tex
%% NOVA thesis document file
%%
%% Chapter with introduciton
%%%%%%%%%%%%%%%%%%%%%%%%%%%%%%%%%%%%%%%%%%%%%%%%%%%%%%%%%%%%%%%%%%%
\newcommand{\novathesis}{\emph{novathesis}}
\newcommand{\novathesisclass}{\texttt{novathesis.cls}}



\chapter{Background and Motivation}%
\label{cha:bg_and_motivation}

\section{Context}%
\label{sec:context}

The idea behind this thesis was born in 2015, at NGNS-IS (a Portuguese
tech startup). At the time, the company's flagship product was the
Forest Fire Finder (\gls{FFF}). The \gls{FFF} was a forest fire
detection system, capable of mostly autonomous and automatic operation.
The system was the first application of Differential Optical Absorption
Spectroscopy~\gls{DOAS} for fire detection, and for that it was patented
in 2007 (see~\cite{Vieira2007, Application2008}). The \gls{FFF} is a
remote sensing device that scans the horizon for the presence of a smoke
column, sequentially performing a chemical analysis of each azimuth,
using the Sun as a light source for its spectroscopic
operations~\cite{ValentedeAlmeida2017}.

The \gls{FFF} was deployed in several "habitats", both nationally
(Parque Nacional da Peneda-Gerês and Ourém) and internationally (Spain
and Brazil). One of the company's clients at the time was interested in
a pollution monitoring solution, and asked if the spectroscopic system
would be capable of performing such a task. The challenge resonated
through the company's structure and the idea that created this thesis
was born. The team then started reading about the concept of Air
Pollution~\gls{AP} and how both populations and entities were concerned
about it. It became clear that, while there were already several methods
to measure \gls{AP}, there was a clear market drive for the development
of a system that could leverage the large area capabilities of a
\gls{DOAS} device while being able to provide a more spatially resolved
"picture" of the atmospheric status. With this in mind, the company
managed to have the investigation financed through a PT2020 funding
opportunity. This achievement was a clear validation of the project's
goals and of the need there was for a system with the proposed
capabilities. It was, however, not enough. \gls{FFF} was a very good
starting point, but there was still a lot of continuous research work
needed before any of the goals that had been set were achieved. This led
to the publication of this PhD project, in a tripartite consortium
between FCT-NOVA, NGNS-IS and the Portuguese Foundation for Science and
Technology. Its main goal was to develop an atmospheric monitoring
system prototype that would be able to spectroscopically map pollutant
concentrations in a two-dimensional way.

\section{The Problem}%
\label{sec:the_problem}

The first step in tackling the development of the proposed system was to
understand the problem it should be dealing with. Air Pollution
(\gls{AP}) is one of the most present concerns of people around the
industrialized modern world. In Europe, it is perceived to be the second
most important threat to the environment. The first is climate change,
which is great part caused by \gls{AP}. Scientists in many countries
have established it as a major cause for premature death, disease onset
and hospital visits for some decades now. Regulatory bodies of many
countries have been gathered to put some legislative pressure on
industries and on society itself, in order to produce a decrease in the
amount of Air Pollution to which people are exposed. These policies and
measures have had a dramatic influence in air quality, which is very
significantly improved throughout the years. In spite of this, official
reports continue to highlight the importance of keeping a vigilant eye
towards \gls{AP} proliferation and its possible undiscovered health
effects. A better introduction to the subject of \gls{AP} is produced in
Section~\ref{sec:problem_introduction}.

Our spectacular progress as a species in the last few centuries has had
some unforeseen adverse consequences. Mitigating them is not only a
responsibility, but also a necessity. In some regards, as is the
\gls{AP} case, this mitigation is only achievable with intelligent and
effective action. This of course demands that we understand, trace and
measure it in as many ways as possible. This project aimed to create
just that: another way in which to measure and map the behavior of
certain pollutants.

\section{Objectives}%
\label{sec:objectives}

From the beginning of the project, the main objective has always been to
design and develop a miniaturized spectroscopic environmental analysis
device. The system would have to be small and portable enough to adapt
to a drone, but should be able to function if adapted to any other
surface, such as a car or even a fixed station. With time, however, the
goal changed somewhat. The miniaturization and drone-adaptation were
kept, but there was now the need to be able to map the pollutant
concentration along a geographic region through the use of tomographic
reconstruction algorithms. This meant that, while the device itself
would be perfectly capable of operating without a drone, the tomographic
capability would be lost, unless a prohibitively large number of
spectroscopic systems were deployed in that region.

The concept evolved in such a way that the  overarching goal of this
thesis is now to theorize and design a two-dimensional mapping tool for
trace atmospheric pollutants such as NO$_x$ and SO$_x$, using passive
DOAS as the measurement technique. In addition to this, it should be
mentioned that the system must be small and portable; use tomographic
reconstruction (and acquisition) algorithms to map the region; and that
it should use only one collection point (to reduce costs and
instrumental complexity). From these objectives, the research questions,
detailed in Section~\ref{sec:research_question} were derived.

% \section{Methods}%
% \label{sec:methods}

% To address these goals, I assumed the development of this thesis to be
% essentially split in two parts, which are to be explored simultaneously.
% They can (coarsely) be addressed as \emph{tomography} and
% \emph{instrumentation}. On the tomography side, it will be necessary to
% study what are the more appropriate algorithms (and what type of
% tomography), how they can be physically deployed (i.e., the problem's
% geometry) and what type of reconstruction method is the more favorable.
% Regarding the instrumentation, there are also several points that need
% considering: decisions are required with regard to the mechanics, the
% controls and the optical components of the final system. A more detailed
% discussion of these topics can be found in
% Section~\ref{sec:hypothesis_and_approach} and
% Chapter~\ref{cha:research_methods}.

% One of the most important steps in the development of this work is
% designing and implementing a simulation software platform, that allows
% the validation of the acquisition strategy, the geometry selection and
% the reconstruction approach. This endeavor will also be of critical
% importance for component selection, since it will define the component
% requirements for the whole system. On the optical instrumentation side,
% this project will require optimizing \gls{FFF}'s optical assembly.
% Although similar in purpose and types of components, this assembly is
% significantly larger than what is acceptable for this project and needs
% redesigning.
