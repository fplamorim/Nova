%!TEX root=../slr.tex

\section{Methods}\label{sec:methods}

An MS, as this article intends to be, always aims to answer its research
questions in a broad but definite way. It is a way of understanding a
given field of research, and being able to systematise how this
understanding is achieved. In this case, the overarching goal of this
study was to provide a solid landscape for the current state of the art
in the field of atmospheric tomography using DOAS as the projection
collection method.

\subsection{Research Questions}
\label{sub:research_questions_and_search_queries}

Before delving into any kind of literary search, we have used the
PICOC(Population, Intervention, Context, Outcome and Comparison) method
for structuring our goals and defining our research questions. This is
summarised in Table~\ref{tab:picoc}, and led us to our research goal:
\textit{assess research status for the DOAS tomography technique, and
identify new investigative approaches}. This paper's research question
aims to provide the foundation for our research goal: \textbf{what is
the current status of the technology used in tomographic DOAS?}
    
\begin{table}[htb]
\small
\centering
\caption{PICOC analysis.}
\label{tab:picoc}
\begin{tabular}{@{}ll@{}}
\toprule
\textbf{Population} & DOAS research in general.\\ 
\textbf{Intervention} & The papers must address tomographic DOAS.\\
\textbf{Outcome} & Status \textbf{assessment} for \textbf{DOAS tomography}.\\
\textbf{Context} & Research papers.\\\bottomrule
\end{tabular}
\end{table}

The mentioned research question is too vague to pursue in a systematic
fashion, so we had to slice it into smaller and more objective chunks.
This sectioning is presented in Table~\ref{tab:rq_slicing}.

\begin{table}[htb]
\centering
\small
\caption{Research question slicing}
\label{tab:rq_slicing}
    \begin{tabularx}{\textwidth}{lX}
        \toprule
        \textbf{Original} & What is the current status of the technology used in
        tomographic DOAS? \\
        \textbf{RQ1} & Is there a typical hardware setup used in tomographic
        DOAS studies? \\
        \textbf{RQ2} & Is there a standard software used to perform these
        analysis? \\
        \textbf{RQ3} & What are the algorithms more commonly used?\\\bottomrule
    \end{tabularx}
\end{table}

The research question is one of the most important steps in planning a
Systematic Literature Review, but it cannot be entered into a library's
search box. Therefore, we have to define our search terms before we can
make any effort of answering our questions.

\subsection{Search Query Definition, Library Selection and Filter Definition}
\label{sub:library_selection_and_filter_definition}

Preliminary searches, ran as tests for this study, indicated that there
were a very low number of studies in DOAS tomography, in comparison to
other subjects to which this methodology is commonly applied. As
consequence, the search terms were purposefully chosen as to maintain a
broad scope and retrieve the largest possible amount of articles. The
selected terms were: \textbf{DOAS atmospher* tomography}\footnote{The
asterisk acts as a wildcard.}. The same strategy was applied to the
selection of electronic libraries, as these preliminary searches
revealed that there was a poor availability of relevant information in
the most commonly used libraries. This in itself motivated a two stage
approach to the search effort, in which we use several libraries to
complement an initial Google Scholar search. Libraries used are
summarised in Table~\ref{tab:libraries}.

\begin{table}[htb]
\centering
\caption{Electronic libraries used in this study.}
\label{tab:libraries}
    \begin{tabularx}{\textwidth}{ll}
        \toprule
        \textbf{Library}          & \textbf{URL}\\
        \midrule
        Google Scholar (GS)   & https://scholar.google.com/\\
        Web of Knowledge (WoK)& https://webofknowledge.com\\
        Scopus (SD)   & https://www.scopus.com\\
        % IEEE             & http://ieeexplore.ieee.org/\\
        % AGU Publications (AGU) & http://agupubs.onlinelibrary.wiley.com/hub/\\
        \bottomrule
    \end{tabularx}
\end{table}

In addition to the search libraries, the study's results will be heavily
influenced by the inclusion and exclusion filters that we apply to the
search results. There are 3 Inclusion Criteria (IC) and one Exclusion
Criteria (EC). The IC determined that selected documents should be
journal papers written in English and be fundamentally about DOAS
tomography. The Exclusion Criteria determined that selected papers do
not include studies about satellite data. These criteria are reflected
in Table~\ref{tab:select_filters}.

\begin{table}[htb]
\centering
\caption{Selection filters in use for this study's search.}
\label{tab:select_filters}
\begin{tabularx}{\textwidth}{lXl}%{@{}cll@{}}
\toprule
\multicolumn{1}{l}{} & \textbf{Criterium} & \textbf{Definition} \\ \midrule
\multirow{1}{*}{\textbf{Exc. Criteria}} & EC1 & Satellite data papers
are not accepted \\
\midrule
\multicolumn{1}{l}{\multirow{3}{*}{\textbf{Inc. Criteria}}} & IC1 &
Results must be journal papers \\
\multicolumn{1}{l}{} & IC2 & Results must be about Tomographic DOAS \\ 
\multicolumn{1}{l}{} & IC3 & Results must be written in English \\ 
\bottomrule
\end{tabularx}
\end{table}

\subsection{Data Extraction Strategy}
\label{sub:data_extraction_strategy}

The data extraction process is a key part of any systematic review. That
being the case, it is also true that there is no "one size fits all"
solution, and that this is probably the part of the study where scope
adaptation is more important. With this in mind, we started defining our
data extraction strategy during the preliminary searches. As already
stated, these searches demonstrated that we would not have many papers
to work with. Moreover, only one element of the team had DOAS expertise.
This led to a data extraction strategy that was comprised of two stages:
in the first stage, the DOAS expert would run the search and analyse
the results. The second stage would see the other team members reading
the analysed articles, ensuring all the inclusion and exclusion criteria
were being correctly applied. Analysis control was conducted using a
shared spreadsheet created in Microsoft Excel.
% End of data extraction strategy

\subsection{Quality Assessment}
\label{sub:quality_assessment}

Assessing an article's quality is not a trivial matter. Normally, there
is almost an immediate association between the term quality and which
paper is "better or worse". This implicit thinking has a subjective
flavour to it that is not welcome in the kind of analysis one needs in a
Systematic Mapping Study. To eliminate this problem, researchers usually
choose some kind of metric of objective nature and rate the various
papers in accordance to this metric. In our case, we have chosen to
evaluate the citation number for each one of the selected articles,
their age and the general quality of the publishing method that was
used. The formula used in this calculation takes the form of
Equation~\ref{eq:score}. In this equation, $Q_{i}$ is the journal's
quartile, and $C_{i}$ is the number of citations of each particular
article. The $Age_i$ factor in the formula is used for relaxing the
citation criteria, according to the age of the article, which will act
as a normalising factor in the number of citations, with the logical
reasoning that older articles will naturally have more citations.

\begin{equation}
    \centering
    \label{eq:score}
    S = Q_{i} \cdot \frac{C_{i}}{Age_{i}}
\end{equation}
