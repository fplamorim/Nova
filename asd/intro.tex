%!TEX root=../slr.tex
\section{Introduction}
\label{sec:introduction}

Differential Optical Absorption Spectroscopy (DOAS) is one of the most
prominent methods for analysing and quantifying atmospheric chemistry,
namely in what concerns trace gas concentrations. The technique,
developed during the 70s by Perner and Platt~\cite{Perner1976}, was
popularised in the following decades by its use in detecting Ozone,
Nitrogen Oxydes and studies of cloud radiative transport. DOAS is a type
of absorption spectroscopy, which uses a clever mathematical and
physical observation to overcome the difficulties of spectral
measurement in the open atmosphere.

Through the setting of very careful geometric considerations, it is
possible to combine DOAS with tomographic reconstruction methods in
order to assemble a map of the gaseous concentrations in a given
geographic region. Tomography is the process of reconstructing an image
through projections obtained by subjecting a given target (in our case,
the atmosphere) to being traversed by any kind of penetrating or
reflecting wave, which in our case is visible light.

With this study, we have intended to capture the current literary
landscape surrounding the usage of tomographic DOAS, assessing this
technique's technological status. For this purpose, we have employed a
review methodology originary from Evidence Based Medicine. This method,
which has migrated to engineering through Software Engineering, is
called a Systematic Mapping Study (MS). It provides a framework that
allows researchers to produce detailed and systematic search protocols,
which are used to catalogue literature information and identify research
gaps within a determined subject.

The search procedure that we have defined was carefully engineered to
cover all tomographic DOAS research relevant to urban, rural or
industrial environments. Through it, we were able to find several
different applications, all pertaining to scientific research, which are
similar regarding physical principle, but differ in equipment assembly,
algorithms, software, and geometry.

The rest of this paper has the following structure:
Section~\ref{sec:background} presents the context within which this
study was written; Section~\ref{sec:methods} describes how we have
planned to perform the study and the methods we have used in doing it;
Section~\ref{sec:conduction} describes the application of the said
methods in the pursuit of our goals and presents the results we have
obtained, as well as our evaluation of our processes;
Section~\ref{sec:conclusions} shows our conclusions and what we think
might be retained from reading this paper.
